\subsubsection{Суть игрового процесса}
    Игровой процесс в данной игре представляет собой захватывающее сочетание платформера и стратегии, где игроку предстоит пройти через пять уникальных уровней, каждый из которых соответствует различным аспектам учебной программы «Компьютерные науки и технологии». Игрок берет на себя роль студента, который сталкивается с агрессивными врагами, представляющими собой дисциплины, предметы и приложения, стремящиеся помешать его успеваемости.\par
    Основные элементы игрового процесса:\par
    \begin{enumerate}
    \item \textbf{Платформенные элементы} \par
    Игроку предстоит прыгать, избегать препятствий и сражаться с врагами, используя свою ловкость и внимательность. Прыжки сверху на врагов — ключевой способ их уничтожения.
    \item\textbf{Стратегия и выбор пути} \par
    Игрок должен разрабатывать свою стратегию на каждом уровне, выбирая наиболее эффективный путь для достижения цели. Это включает в себя анализ препятствий и врагов, а также принятие решений о том, как лучше всего использовать свои навыки.
    \item \textbf{Система оценок} \par
    За успешное преодоление уровней и уничтожение врагов игрок получает «десятки», которые служат не только для прохождения на следующий уровень, но и как символ успеха и знаний игрока.
    \item \textbf{Разнообразие врагов и препятствий} \par
    Каждый уровень предлагает уникальных врагов и препятствий, что делает игровой процесс разнообразным и увлекательным. Игрок сталкивается с различными тактиками и подходами для победы над каждым видом противника.
    \end{enumerate}
    Игрок получает удовольствие от сочетания динамичного игрового процесса, стратегического мышления и возможности видеть свой прогресс, переходя с одного уровня на другой. Игра не только развлекает, но и позволяет глубже понять трудности и радости студенческой жизни.

\subsubsection{Ход игры и сюжет}
    \\Игровой сеанс начинается с того, что игрок погружается в мир Высшей Школы Экономики, где он играет за студента, который оказался в странной реальности, где учебные дисциплины и предметы стали врагами. Сюжет игры разворачивается через пять уровней, каждый из которых представляет собой отдельный курс обучения.\par
    В финале игрок сражается с дипломом, который использует все механики из предыдущих уровней. Эта битва требует от игрока максимальной концентрации, стратегического мышления и применения всех навыков, которые он приобрел на протяжении игры.\par
    После победы над дипломом игрок получает видит, как его знания и навыки применяются в реальной жизни на самом деле. Концовка игры заставляет задуматься о пройденном пути, подчеркивая важность образования и преодоления трудностей, с которыми сталкиваются студенты.
    Игрок выходит из игры с чувством удовлетворения и понимания, что обучение — это не только трудности, но и достижения.