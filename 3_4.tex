\subsection*{3.4. Элементы игры}

Игра представляет из себя платформер с элементами roguelike — это жанр компьютерных игр, который сочетает в себе элементы двух  жанров: платформера и roguelike.  

\begin{enumerate}
	\item \textbf{Противники} \par
	На каждом уровне игрок сталкивается с уникальными врагами, которые требуют разных подходов для победы над ними.
	Противники подразделяются на классы:
	\begin{itemize}
		\item \textbf{Противники ближнего боя:} \par
			Итеграл, философия
		\item \textbf{Противники дальнего боя:} \par
			SmartLMS, python, ворона, Яндекс.Почта
        \item \textbf{Боссы:} \par
            Диплом
	\end{itemize}
   Каждый из этих противников олицетворяет собой одну из трудностей, с которыми сталкиваются студенты Высшей школы экономики на пути к выпуску.
    \item \textbf{Уровни} \par
 Каждый уровень представляет собой отдельный курс обучения. Игра состоит из пяти уровней, и на каждом из них игрока ждут разнообразные противники и уникальное расположение игровых элементов.
    \item \textbf{Игровые элементы} \par
    \begin{itemize}
		\item Препятствия, ловушки
		\item Платформы, являющиеся основной составляющей уровней.
	\end{itemize}
 \item \textbf{Сложность} \par
 С каждым уровнем сложность игры повышается, а поведение врагов становится более непредсказуемым. Это заставляет игрока постоянно осваивать новые игровые механики, что поддерживает интерес к игре. 
 \item \textbf{Награды и достижения} \par
   За каждый пройденный уровень игрок получает достижение, которое можно сравнить с высшей оценкой в университетской шкале — «10». 
    
\end{enumerate}
