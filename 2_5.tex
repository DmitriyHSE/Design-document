\item Предпосылки создания (общие тенденции рынка в данном направлении)
\begin{enumerate}
    \item \textbf{Ностальгия} \par
    Многие игроки, особенно представители поколения, выросшего на классических играх 90-х и начала 2000-х, испытывают ностальгию по 2D-платформерам. Игры этого жанра, такие как \textbf{Super Mario Bros., Sonic the Hedgehog, Castlevania}, стали культовыми и стали важной частью истории видеоигр. В современном мире разработчики активно используют ретро-стилистику, а также создают множество ремастеров и ремейков, чтобы привлечь внимание как старых фанатов жанра, так и новых игроков, которые хотят познакомиться с классическим игровым геймплеем.
    \item\textbf{Быстрая и простая разработка} \par
    Разработка 2D-игр требует значительн меньше времени и усилий, из-за чего остается много ресурсов для создания уникальных механик и экспериментов с ними. Это делает 2D-платформеры привлекательными для инди-разработчиков и небольших студий, у которых, в основном, есть идеи, но нет денежных средств. Это приводит к высокому уровню разнообразия и творческого подхода в инди-сцене.
    \item \textbf{Фокус на геймплей и механики} \par
    2D-платформеры, как правило, больше акцентируют внимание на геймплейных механиках, чем на графических аспектах или глубоком сюжете. В таких играх важна точность управления, баланс сложности и хорошо продуманные уровни — это делает их привлекательными для игроков, которым важен именно игровой процесс. В отличие от многих современных AAA-проектов, которые сосредоточены на кинематографичности, 2D-платформеры возвращают игрока к основам видеоигр, где ключевым аспектом является взаимодействие с миром и решение поставленных задач. Примеры успешных игр, таких как \textbf{Hollow Knight} или \textbf{Dead Cells}, показывают, что хорошо продуманный геймпей может привлечь внимание широкой аудитории.
    \item \textbf{Простая адаптация под мобильные устройства} \par
    Мобильные платформы, такие как смартфоны и планшеты, также способствуют популярности 2D-платформеров. Управление в таких играх проще адаптировать для сенсорных экранов, чем 3D-игры с камерой и движением в пространстве, что также влияет на удобность самого управления.
\end{enumerate}