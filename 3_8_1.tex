\subsubsection{Общее описание}
    \begin{enumerate}
    \item \textbf{Техническое исполнение} \par
    Графика игры выполнена в 2D формате. В игре детально проработаны модели и анимации персонажей, так как каждый элемент игры отрисован вручную. Использование векторной графики позволило сделать линии более чёткими и масштабировать модели без потери качества.
    \item\textbf{Стилистика, атмосфера и палитра} \par
    Графический стиль игры базируется на карикатурной идее, поэтому игра ощущается легко и позитивно. Яркий и насыщенный белый фон влекёт за собою внимание и погружает игрока в живую среду студента Вышки. А на контрасте с тёмным кирпичом даёт понять игроку, что путь его ждёт не простой. Анимации персонажей усиливают чувство динамичности игры.
    \item \textbf{Другие общие сведения} \par
    Графика в игре сочетает в себе элементы платформера и стратегии. Сочетание платформера и стратегии в графическом исполнении однозначно придаёт проекту уникальности и разнообразия. Визуальные элементы платформера, такие как блочные структуры и препятствия, чётко выделены и легко воспринимаются. Это позволяет игрокам быстро ориентироваться в пространстве и строить стратегии для преодоления препятствий. Анимации врагов и персонажей проработаны таким образом, чтобы игрок мог предугадывать их действия. Динамичность анимации делает игру более захватывающей — сражения с врагами превращаются в живое представление.
    \end{enumerate}