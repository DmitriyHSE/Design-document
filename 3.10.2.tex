\documentclass{article}
\usepackage[utf8]{inputenc}
\usepackage[russian]{babel}
\usepackage{amsmath}
\usepackage{graphicx}
\begin{document}

\subsection*{3.10.2}

\begin{itemize}
    \item \textbf{Диаграмма взаимного расположения уровней} \\
    В данной игре уровни представляют собой последовательные курсы обучения в ВШЭ, расположенные по линейно-последовательной схеме, где каждый уровень содержит свои уникальные враги, усложняющие жизнь студентов. Успешное преодоление всех уровней ведет к ключевой локации - пятому уровню - финальному сражению с дипломом.
    Игра содержит в себе следующие уровни:
    \begin{itemize}
        \item Первый курс
        \\Враги: Оценка "-10", интеграл, Smart LMS
        \item Второй курс
        \\Враги: Английский язык, Python, теория вероятностей и математическая статистика
        \item Третий курс
        \\Враги: Machine Learning, Яндекс Контест, курсовая работа
        \item Четвёртый курс
        \\Враги: Философия, значок Яндекс Почты с непрочитанными письмами, вышкинская ворона
        \item Финальный уровень
       \\ Босс: Диплом
    \end{itemize}
    Каждый уровень увеличивает сложность и добавляет новые вызовы на пути студента, что создает интересный игровой процесс.
\end{itemize}

\end{document}