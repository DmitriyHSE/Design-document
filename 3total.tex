\documentclass{article}
\usepackage[utf8]{inputenc}
\usepackage[russian]{babel}
\usepackage{amsmath}

\begin{document}

\section{Часть 1}

\subsubsection{Суть игрового процесса}
Игровой процесс в данной игре представляет собой захватывающее сочетание платформера и стратегии, где игроку предстоит пройти через пять уникальных уровней, каждый из которых соответствует различным аспектам учебной программы «Компьютерные науки и технологии». Игрок берет на себя роль студента, который сталкивается с агрессивными врагами, представляющими собой дисциплины, предметы и приложения, стремящиеся помешать его успеваемости.\par
Основные элементы игрового процесса:\par
\begin{enumerate}
    \item \textbf{Платформенные элементы} \par
    Игроку предстоит прыгать, избегать препятствий и сражаться с врагами, используя свою ловкость и внимательность. Прыжки сверху на врагов — ключевой способ их уничтожения.
    \item \textbf{Стратегия и выбор пути} \par
    Игрок должен разрабатывать свою стратегию на каждом уровне, выбирая наиболее эффективный путь для достижения цели. Это включает в себя анализ препятствий и врагов, а также принятие решений о том, как лучше всего использовать свои навыки.
    \item \textbf{Система оценок} \par
    За успешное преодоление уровней и уничтожение врагов игрок получает «десятки», которые служат не только для прохождения на следующий уровень, но и как символ успеха и знаний игрока.
    \item \textbf{Разнообразие врагов и препятствий} \par
    Каждый уровень предлагает уникальных врагов и препятствий, что делает игровой процесс разнообразным и увлекательным. Игрок сталкивается с различными тактиками и подходами для победы над каждым видом противника.
\end{enumerate}
Игрок получает удовольствие от сочетания динамичного игрового процесса, стратегического мышления и возможности видеть свой прогресс.

\section{Часть 2}

\subsection*{Персонаж игрока}

\begin{itemize}
    \item \textbf{Персонаж игрока - студент Высшей школы экономики} \\ 
    Данный персонаж - добрый, замотированный в учебе и преодолении различных сложностей во время учебного процесса студент. Он готов справляться с ними во что бы то ни стало и не опускать руки. На персонаже футболка из мерча Высшей школы экономики с надписью "HSE" для поддержания корпоративного стиля ВУЗа. В качестве цвета футболки выбран синий - самый узнаваемый цвет из брендбука ВШЭ, с которым знаком каждый студент Вышки. 
\end{itemize}

\section{Часть 3}

\subsection*{Элементы игры}

Игра представляет из себя платформер с элементами roguelike — это жанр компьютерных игр, который сочетает в себе элементы двух жанров: платформера и roguelike.

\begin{enumerate}
    \item \textbf{Противники} \par
    На каждом уровне игрок сталкивается с уникальными врагами, которые требуют разных подходов для победы над ними.
    Противники подразделяются на классы:
    \begin{itemize}
        \item \textbf{Противники ближнего боя:} \par
        Итеграл, философия
        \item \textbf{Противники дальнего боя:} \par
        SmartLMS, python, ворона, Яндекс.Почта
        \item \textbf{Боссы:} \par
        Диплом
    \end{itemize}
    Каждый из этих противников олицетворяет собой одну из трудностей, с которыми сталкиваются студенты Высшей школы экономики на пути к выпуску.
    \item \textbf{Уровни} \par
    Каждый уровень представляет собой отдельный курс обучения. Игра состоит из пяти уровней, и на каждом из них игрока ждут разнообразные противники и уникальное расположение игровых элементов.
    \item \textbf{Игровые элементы} \par
    \begin{itemize}
        \item Препятствия, ловушки
        \item Платформы, являющиеся основной составляющей уровней.
    \end{itemize}
    \item \textbf{Сложность} \par
    С каждым уровнем сложность игры повышается, а поведение врагов становится более непредсказуемым. Это заставляет игрока постоянно осваивать новые игровые механики, что поддерживает интерес к игре.
\end{enumerate}

\section{Часть 4}

\subsection*{Физическая модель}

\begin{itemize}
    \item \textbf{Физическая модель игрового мира} \par
    Физическая модель игрового мира базируется на упрощенной физике, которая делает управление персонажем интуитивно понятным и отзывчивым для игрока. Основные законы физики такие, как гравитация и т.п., адаптированы для создания ощущения плавного и предсказуемого взаимодействия между персонажем, объектами и окружением.
    \item \textbf{Перемещение} \par
    Персонаж может перемещаться влево или вправо с постоянной максимальной скоростью, достигая её плавно (ускорение). Также, как и почти во всех играх такого типа, реализована механика прыжков, которые подчиняются параболической траектории и высотой которых можно управлять с помощью зажатия клавиши прыжка.
    \item \textbf{Боевые действия} \par
    Прыжок сверху на врага — это базовая атака, которая уничтожает или обезвреживает большинство врагов. Это ключевая механика, которая является простой для освоения, но при этом может быть достаточно сложной в реализации для определенных противников.
\end{itemize}

\section{Часть 5}

\subsection*{Искусственный интеллект}

Искусственный интеллект (AI) в игре разработан так, чтобы создавать атмосферу, отражающую реальные студенческие трудности, при этом оставаясь простым и понятным для игрока.

\textbf{Общие принципы поведения AI:}

\begin{enumerate}
    \item \textbf{Противники ближнего боя:}  
    Эти враги, такие как интеграл или значок Яндекс.Почты, патрулируют ограниченные территории. Если игрок попадает в зону их действия, они начинают преследование. При уходе игрока за пределы зоны враги возвращаются к своему маршруту.  

    \item \textbf{Противники дальнего боя:}  
    Например, <<-10>> или философия. Эти враги атакуют, если игрок оказывается в их зоне видимости. Они фиксируются на игроке до тех пор, пока он не укрывается или не выходит за пределы зоны атаки.  

    \item \textbf{Летающие противники:}  
    Такие как вышкинская ворона или SmartLMS. Эти враги двигаются по заданной траектории, периодически стреляя в игрока. Если игрок долго находится в их зоне, враги могут начать его преследовать.  

    \item \textbf{Поведение на уровнях:}  
    \begin{itemize}
        \item На первых уровнях враги действуют просто: патрулируют или атакуют игрока при входе в зону действия.  
        \item На более сложных уровнях враги могут действовать координированно, усиливая давление на игрока. Например, ближние и дальние противники могут перекрывать пути отхода.  
    \end{itemize}
    \item \textbf{Эмоциональный эффект:}  
    \begin{itemize}
        \item AI создаёт умеренный вызов для игрока, поддерживая атмосферу игры.  
        \item Противники не только атакуют, но и добавляют комичные элементы, например, ироничные фразы или необычное поведение, что делает игровой процесс более увлекательным.  
    \end{itemize}
\end{enumerate}

\end{document}
