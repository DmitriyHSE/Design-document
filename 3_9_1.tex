\subsubsection{Общее описание}
    \begin{enumerate}
    Музыкальные темы и звуковые эффекты Высшей Школы Экономики отражают звуковой ландшафт, который усиливает игровой процесс и передаёт дух учебного процесса студентов.
    \item \textbf{Характер музыкальных тем} \par
    Каждый уровень игры приурочен к определенному курсу обучения, поэтому музыкальное сопровождение варьируется в зависимости от темы и настроения, соответствующего выбранному курсу. Например, для первого уровня, который может представлять собой вводный курс, музыка будет легкой и мелодичной, с элементами оптимизма и надежды. По мере продвижения к более сложным курсам, музыкальные темы становятся более напряжёнными и сложными. На втором и третьем уровнях звуки производят ощущение давления испытываемое студентами.
    \item\textbf{Создаваемое настроение} \par
    С каждой новой темой углубляется и эмоциональная нагрузка игры. Четвертый уровень, где студенты сталкиваются с более серьезными испытаниями, сопровождён мрачной и загадочной музыкой. Здесь игрок может ощутить важность своих решений, когда каждая ошибка может стать фатальной. Финальный уровень, посвященный защите диплома, будет являться кульминацией, а потому здесь присутствуют эпические звучания и динамичные ритмов. Перед лицом диплома, музыка поднимется до максимумальной напряжённости.
    \item \textbf{Насыщенность мира звуками} \par
    Звуковая палитра игры обогащена многочисленными звуковыми эффектами: шаги по коридорам университета, оплеухи врагов и победы над ними. В каждом уровне используются свои уникальные звуки. Отдельное внимание уделяется эффектам, возникающим в моменты столкновения с врагами, например, различные звуки для «попадания» и «неудачи».
    \end{enumerate}