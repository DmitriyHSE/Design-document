\documentclass{article}
\usepackage[utf8]{inputenc}
\usepackage[russian]{babel}
\usepackage{amsmath}

\begin{document}

\subsection*{Введение}

\begin{enumerate}
    \item \textbf{Комментарии по организации содержимого документа.} \\
    Данный дизайн-документ будет составлен по лабораторной работе №1 на Unity. Более подробную информацию об этом проекте вы сможете узнать в этом документе в разделе «Концепция».

    \item \textbf{Ссылки на и используемые материалы, копирайты и прочее.} \\
    Github репозиторий нашего проекта на Unity: https://clck.ru/3EfV6j.

    \item \textbf{История изменений документа.} \\
    В процессе разработки дизайн документа история изменений будет формироваться и дополняться.

    \item \textbf{Список авторов.} \\
    1) Дмитрий Цирулёв\\2) Анастасия Шепшелюк\\3) Егор Гусев\\4) Николай Сапрыкин\\5) Леонид Высоцкий\\6) Никита Троицкий

    \item \textbf{Условные обозначения, сокращения и другие соглашения.} \\
    Сокращение: «Компьютерные науки и технологии» - КНТ.

    \item \textbf{Любые сведения, которые необходимы для прочтения документа.} \\
    Рекомендуемое знание русского языка и желательное знание редактора на основе LaTeX.
\end{enumerate}

\end{document}