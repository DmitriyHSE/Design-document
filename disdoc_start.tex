\documentclass{article}
\usepackage[T2A]{fontenc}
\usepackage{authblk}

\title{Дизайн-документ игры \\"Вышка: RogueLike Adventure"}
\author{Дмитрий Цирулёв, Анастасия Шепшелюк, Егор Гусев, Николай Сапрыкин, Леонид Высоцкий, Никита Троицкий }



\date{Ноябрь 2024}

\begin{document}

\maketitle

\tableofcontents
\newpage
\section{Введение}
Some text about the game which is introduction
\newpage
\section{Концепция}
\begin{itemize}
    \item Введение
    \item Жанр и аудитория
    \begin{itemize}
        \item Жанр \par
        \textbf{Roguelike} - поджанр ролевых игр, характеризующийся процедурно генерируемыми уровнями, пошаговым игровым процессом и элементами постоянной смерти. Ключевые особенности игры - это пошаговый игровой процесс, постоянная смерть, процедурно генерируемые уровни и исследование. 
        \item Возрастная категория \par
        \textbf{Аудитория} - 18+. Это возрастное ограничение выбрано для большего понимания философии игры, так как целевая аудитория пользоваелей - студенты, в особенности студенты Высшей школы экономики. В большинстве случаев студенты ВШЭ находятся именно в этом возрастном диапазоне.
    \end{itemize}
    \item Основные особенности игры
    \item Описание игры
    \item Предпосылки создания
    \item Платформа
\end{itemize}
\subsection{Введение}

\subsection{Жанр и аудитория}

\subsection{Основные особенности игры}

\subsection{Описание игры}

\subsection{Предпосылки создания}

\subsection{Платформа}

\section{Функциональная спецификация}

\subsection{Принципы игры}

\subsubsection{Суть игрового процесса}

\subsubsection{Ход игры и сюжет}

\subsection{Физическая модель}

\subsection{Персонаж игрока}

\subsection{Элементы игры}

\subsection{«Искусственный интеллект»}

\subsection{Многопользовательский режим}

\subsection{Интерфейс пользователя}

\subsubsection{Блок-схема}

\subsubsection{Функциональное описание и управление}

\subsubsection{Объекты интерфейса пользователя}

\subsection{Графика и видео}

\subsubsection{Общее описание}

\subsubsection{Двумерная графика и анимация}

\subsubsection{Трехмерная графика и анимация}

\subsubsection{Анимационные вставки}

\subsection{Звуки и музыка}

\subsubsection{Общее описание}

\subsubsection{Звук и звуковые эффекты}

\subsubsection{Музыка}

\subsection{Описание уровней}

\subsubsection{Общее описание дизайна уровней}

\subsubsection{Диаграмма взаимного расположения уровней}

\subsubsection{График введения новых объектов}

\section{Контакты}


\end{document}