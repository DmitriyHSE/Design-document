\documentclass{article}
\usepackage[T2A]{fontenc}
\usepackage{authblk}

\title{Дизайн-документ игры \\"Вышка: RogueLike Adventure"}
\author{Дмитрий Цирулёв, Анастасия Шепшелюк, Егор Гусев, Николай Сапрыкин, Леонид Высоцкий, Никита Троицкий }



\date{Ноябрь 2024}

\begin{document}

\maketitle

\tableofcontents
\newpage
\section{Введение}
Some text about the game which is introduction
\newpage
\section{Концепция}
\begin{itemize}
    \item Введение \par
      Игра представляет собой повествование о непростом пути студента Высшей школы экономики, стремящегося к получению диплома. Каждый уровень игры соответствует определённому году обучения, и на каждом из них героя ждут разнообразные и уникальные противники. По достижении последнего года обучения герой сталкивается с главным противником, в университетских кругах, которого называют «Диплом», имя которого с трепетом и ужасом вспоминают все студенты, ведь от результата этого сражения зависит их дальнейшая судьба. В случае неудачи герой, следуя традициям игр жанра roguelike, вынужден начать свой путь заново, чтобы вновь попытаться преодолеть все трудности и преграды этого сурового мира.
    \item Жанр и аудитория
    \begin{itemize}
        \item Жанр \par
        \textbf{Roguelike} - поджанр ролевых игр, характеризующийся процедурно генерируемыми уровнями, пошаговым игровым процессом и элементами постоянной смерти. Ключевые особенности игры - это пошаговый игровой процесс, постоянная смерть, процедурно генерируемые уровни и исследование. 
        \item Возрастная категория \par
        \textbf{Аудитория} - 18+. Это возрастное ограничение выбрано для большего понимания философии игры, так как целевая аудитория пользоваелей - студенты, в особенности студенты Высшей школы экономики. В большинстве случаев студенты ВШЭ находятся именно в этом возрастном диапазоне.
    \end{itemize}
    \item Основные особенности игры
    \begin{enumerate}
    \item \textbf{Аутентичная атмосфера студенческой жизни КНТ.} \\
    Игра максимально реалистично и с юмором передаёт повседневные трудности, с которыми сталкиваются студенты Высшей школы экономики, особенно направления "Компьютерные науки и технологии". Каждое задание и персонаж основаны на реальных аспектах студенческой жизни, что делает игру близкой и понятной целевой аудитории.

    \item \textbf{Механика прогрессии "от новичка до дипломника".} \\
    Игрок проходит путь от первокурсника до выпускника, сталкиваясь с возрастающими вызовами, уникальными для каждого курса, а также с символическими противниками, отражающими реальные "болевые точки" студентов.

    \item \textbf{Мультижанровый подход к игровому процессу.} \\
    На каждом уровне предлагаются различные игровые механики. Например, сражения с врагами (интегралы и SmartLMS) могут быть выполнены в стиле головоломок, прохождение курса по ML --- в формате мини-контеста, а сражение с дипломом --- в виде эпической битвы с элементами стратегии.

    \item \textbf{Взаимодействие с внутриигровыми NPC.} \\
    Уникальные персонажи-одногруппники и преподаватели помогут (или помешают) игроку в прохождении уровней. Каждый NPC обладает своими характерами, которые можно узнавать по ироничным диалогам.

    \item \textbf{Коллекция мемов и "пасхалок".} \\
    В игру добавлены десятки мемов и пасхалок, которые понятны и близки студентам ВШЭ. От шуточных реплик преподавателей до знаменитой красной кнопки.
    \end{enumerate}

    \textbf{Примерный объём игры:}

    \begin{itemize}
        \item Полное прохождение игры: \textbf{8--10 часов}.
        \item Один уровень (курс): \textbf{1,5--2 часа}, включая все задания и побочные активности.
        \item Босс-файты (особенно защита диплома): \textbf{30--45 минут}, включая подготовку и основную битву.
    \end{itemize}
    \item Описание игры
    \item Предпосылки создания
    \item Платформа
\end{itemize}
\subsection{Введение}

\subsection{Жанр и аудитория}

\subsection{Основные особенности игры}

\subsection{Описание игры}

\subsection{Предпосылки создания}

\subsection{Платформа}

\section{Функциональная спецификация}

\subsection{Принципы игры}

\subsubsection{Суть игрового процесса}

\subsubsection{Ход игры и сюжет}

\subsection{Физическая модель}

\subsection{Персонаж игрока}

\subsection{Элементы игры}

\subsection{«Искусственный интеллект»}

\subsection{Многопользовательский режим}

\subsection{Интерфейс пользователя}

\subsubsection{Блок-схема}

\subsubsection{Функциональное описание и управление}

\subsubsection{Объекты интерфейса пользователя}

\subsection{Графика и видео}

\subsubsection{Общее описание}

\subsubsection{Двумерная графика и анимация}

\subsubsection{Трехмерная графика и анимация}

\subsubsection{Анимационные вставки}

\subsection{Звуки и музыка}

\subsubsection{Общее описание}

\subsubsection{Звук и звуковые эффекты}

\subsubsection{Музыка}

\subsection{Описание уровней}

\subsubsection{Общее описание дизайна уровней}

\subsubsection{Диаграмма взаимного расположения уровней}

\subsubsection{График введения новых объектов}

\section{Контакты}


\end{document}
