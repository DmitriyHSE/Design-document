\documentclass{article}
\usepackage[utf8]{inputenc}
\usepackage[russian]{babel}
\usepackage{amsmath}

\begin{document}

\section{Введение}

\begin{enumerate}
    \item \textbf{Комментарии по организации содержимого документа.} \\
    Данный дизайн-документ будет составлен по лабораторной работе №1 на Unity. Более подробную информацию об этом проекте вы сможете узнать в этом документе в разделе «Концепция».

    \item \textbf{Ссылки на и используемые материалы, копирайты и прочее.} \\
    Github репозиторий нашего проекта на Unity: https://clck.ru/3EfV6j.

    \item \textbf{История изменений документа.} \\
    В процессе разработки дизайн документа история изменений будет формироваться и дополняться.

    \item \textbf{Список авторов.} \\
    1) Дмитрий Цирулёв\\2) Анастасия Шепшелюк\\3) Егор Гусев\\4) Николай Сапрыкин\\5) Леонид Высоцкий\\6) Никита Троицкий

    \item \textbf{Условные обозначения, сокращения и другие соглашения.} \\
    Сокращение: «Компьютерные науки и технологии» - КНТ.

    \item \textbf{Любые сведения, которые необходимы для прочтения документа.} \\
    Рекомендуемое знание русского языка и желательное знание редактора на основе LaTeX.
\end{enumerate}

\newpage
\section{Концепция}
\begin{itemize}
    \item Введение \par
      Игра представляет собой повествование о непростом пути студента Высшей школы экономики, стремящегося к получению диплома. Каждый уровень игры соответствует определённому году обучения, и на каждом из них героя ждут разнообразные и уникальные противники. По достижении последнего года обучения герой сталкивается с главным противником, в университетских кругах, которого называют «Диплом», имя которого с трепетом и ужасом вспоминают все студенты, ведь от результата этого сражения зависит их дальнейшая судьба. В случае неудачи герой, следуя традициям игр жанра roguelike, вынужден начать свой путь заново, чтобы вновь попытаться преодолеть все трудности и преграды этого сурового мира.
    \item Жанр и аудитория
    \begin{itemize}
        \item Жанр \par
        \textbf{Roguelike} - поджанр ролевых игр, характеризующийся процедурно генерируемыми уровнями, пошаговым игровым процессом и элементами постоянной смерти. Ключевые особенности игры - это пошаговый игровой процесс, постоянная смерть, процедурно генерируемые уровни и исследование. 
        \item Возрастная категория \par
        \textbf{Аудитория} - 18+. Это возрастное ограничение выбрано для большего понимания философии игры, так как целевая аудитория пользоваелей - студенты, в особенности студенты Высшей школы экономики. В большинстве случаев студенты ВШЭ находятся именно в этом возрастном диапазоне.
    \end{itemize}
    \item Основные особенности игры
    \begin{enumerate}
    \item \textbf{Аутентичная атмосфера студенческой жизни КНТ.} \\
    Игра максимально реалистично и с юмором передаёт повседневные трудности, с которыми сталкиваются студенты Высшей школы экономики, особенно направления "Компьютерные науки и технологии". Каждое задание и персонаж основаны на реальных аспектах студенческой жизни, что делает игру близкой и понятной целевой аудитории.

    \item \textbf{Механика прогрессии "от новичка до дипломника".} \\
    Игрок проходит путь от первокурсника до выпускника, сталкиваясь с возрастающими вызовами, уникальными для каждого курса, а также с символическими противниками, отражающими реальные "болевые точки" студентов.

    \item \textbf{Мультижанровый подход к игровому процессу.} \\
    На каждом уровне предлагаются различные игровые механики. Например, сражения с врагами (интегралы и SmartLMS) могут быть выполнены в стиле головоломок, прохождение курса по ML --- в формате мини-контеста, а сражение с дипломом --- в виде эпической битвы с элементами стратегии.

    \item \textbf{Взаимодействие с внутриигровыми NPC.} \\
    Уникальные персонажи-одногруппники и преподаватели помогут (или помешают) игроку в прохождении уровней. Каждый NPC обладает своими характерами, которые можно узнавать по ироничным диалогам.

    \item \textbf{Коллекция мемов и "пасхалок".} \\
    В игру добавлены десятки мемов и пасхалок, которые понятны и близки студентам ВШЭ. От шуточных реплик преподавателей до знаменитой красной кнопки.
    \end{enumerate}

    \textbf{Примерный объём игры:}

    \begin{itemize}
        \item Полное прохождение игры: \textbf{8--10 часов}.
        \item Один уровень (курс): \textbf{1,5--2 часа}, включая все задания и побочные активности.
        \item Босс-файты (особенно защита диплома): \textbf{30--45 минут}, включая подготовку и основную битву.
    \end{itemize}
    \item Описание игры

    \begin{enumerate}
    Сюжет игры разворачивается в стенах Высшей Школы Экономики, где главный герой - игрок, являющийся студентом программы «Компьютерные науки и технологии», оказывается в мире, где дисциплины, предметы и приложения превратились в агрессивных врагов, стремящихся помешать его успеваемости и сопутствующих его отчислению.\\  
    \\Цель – выжить и пройти пять уровней, соответствующих каждому из курсов обучения. Каждый уровень – это уникальное испытание, где игрок должен справляться с врагами и получать «десятки», чтобы получить доступ к следующему уровню и к финальной битве за долгожданный диплом.\\
    \\Игровой процесс построен на сочетании разнообразных препятствий. Игроку необходимо преодолеть их, используя собственные ловкость и внимательность. Чтобы уничтожить противника, игроку необходимо прыгнуть на него сверху.\\
    \\Игрок должен разрабатывать стратегию на каждом уровне, выбирая наиболее эффективный путь для достижения цели. Финальный уровень будет представлять из себя битву с боссом. Победа над ним требует применения всех приобретенных навыков управления и стратегического мышления.\\
    \\По окончании игры пользователь поймёт каково же быть студентом программы «Компьютерные науки и технологии». А концовка заставит задуматься о пройденном пути.
    \end{enumerate}
    \item Предпосылки создания
    \begin{enumerate}
    \item \textbf{Ностальгия} \par
    Многие игроки, особенно представители поколения, выросшего на классических играх 90-х и начала 2000-х, испытывают ностальгию по 2D-платформерам. Игры этого жанра, такие как \textbf{Super Mario Bros., Sonic the Hedgehog, Castlevania}, стали культовыми и стали важной частью истории видеоигр. В современном мире разработчики активно используют ретро-стилистику, а также создают множество ремастеров и ремейков, чтобы привлечь внимание как старых фанатов жанра, так и новых игроков, которые хотят познакомиться с классическим игровым геймплеем.
    \item\textbf{Быстрая и простая разработка} \par
    Разработка 2D-игр требует значительн меньше времени и усилий, из-за чего остается много ресурсов для создания уникальных механик и экспериментов с ними. Это делает 2D-платформеры привлекательными для инди-разработчиков и небольших студий, у которых, в основном, есть идеи, но нет денежных средств. Это приводит к высокому уровню разнообразия и творческого подхода в инди-сцене.
    \item \textbf{Фокус на геймплей и механики} \par
    2D-платформеры, как правило, больше акцентируют внимание на геймплейных механиках, чем на графических аспектах или глубоком сюжете. В таких играх важна точность управления, баланс сложности и хорошо продуманные уровни — это делает их привлекательными для игроков, которым важен именно игровой процесс. В отличие от многих современных AAA-проектов, которые сосредоточены на кинематографичности, 2D-платформеры возвращают игрока к основам видеоигр, где ключевым аспектом является взаимодействие с миром и решение поставленных задач. Примеры успешных игр, таких как \textbf{Hollow Knight} или \textbf{Dead Cells}, показывают, что хорошо продуманный геймпей может привлечь внимание широкой аудитории.
    \item \textbf{Простая адаптация под мобильные устройства} \par
    Мобильные платформы, такие как смартфоны и планшеты, также способствуют популярности 2D-платформеров. Управление в таких играх проще адаптировать для сенсорных экранов, чем 3D-игры с камерой и движением в пространстве, что также влияет на удобность самого управления.
    \end{enumerate}
    \item Платформа
    \begin{enumerate}
    \centering
    \caption{Системные требования}
    \end{enumerate}
\end{itemize}
\subsection{Введение}

\subsection{Жанр и аудитория}

\subsection{Основные особенности игры}

\subsection{Описание игры}

\subsection{Предпосылки создания}

\subsection{Платформа}

\section{Функциональная спецификация}

\subsection{Принципы игры}

\subsubsection{Суть игрового процесса}

\subsubsection{Ход игры и сюжет}

\subsection{Физическая модель}

\subsection{Персонаж игрока}

\subsection{Элементы игры}

\subsection{«Искусственный интеллект»}

\subsection{Многопользовательский режим}

\subsection{Интерфейс пользователя}

\subsubsection{Блок-схема}

\subsubsection{Функциональное описание и управление}

\subsubsection{Объекты интерфейса пользователя}

\subsection{Графика и видео}

\subsubsection{Общее описание}

\subsubsection{Двумерная графика и анимация}

\subsubsection{Трехмерная графика и анимация}

\subsubsection{Анимационные вставки}

\subsection{Звуки и музыка}

\subsubsection{Общее описание}

\subsubsection{Звук и звуковые эффекты}

\subsubsection{Музыка}

\subsection{Описание уровней}

\subsubsection{Общее описание дизайна уровней}

\subsubsection{Диаграмма взаимного расположения уровней}

\subsubsection{График введения новых объектов}

\section{Контакты}


\end{document}
