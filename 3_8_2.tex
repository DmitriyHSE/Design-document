\documentclass{article}
\usepackage[utf8]{inputenc}
\usepackage[russian]{babel}

\title{3.8.2}
\author{}
\date{}

\begin{document}

\maketitle

\section{Двумерная графика и анимация}

В данном разделе описана основная 2D графика и анимация, которые необходимо разработать для игры.

\subsection{Интерфейс}
В игре реализовано меню с возможностью начать новую игру или продолжить с того места, где игрок остановился. Также предусмотрена опция выхода из игры с автосохранением. Интерфейс выполнен в минималистичном, но интуитивно понятном стиле. Управление персонажем осуществляется с помощью клавиш \texttt{WASD} или стрелок на клавиатуре, взаимодействие с меню — через курсор мыши.

\subsection{Эффекты}
При уничтожении врагов происходит эффект их мгновенного исчезновения. Также при попадании снарядов противника в игрока, тот моментально выводится из игры, что сопровождается визуальным эффектом.

\subsection{Персонажи, строения и юниты}
Для моделей персонажей, противников и уровней использована векторная графика. Нарисовано: 1 главный герой, 12 различных противников, 1 босс, блоки для генерации уровня и фоновая заставка. Персонаж имеет анимацию передвижения и прыжков.

\subsection{Игровой мир}
Игровой мир выполнен в светлых тонах: стены корпуса Вышки имеют белый цвет, а пол и потолок — тёмные оттенки кирпича.

\end{document}
