\documentclass{article}
\usepackage[utf8]{inputenc}
\usepackage[russian]{babel}
\usepackage{amsmath}

\begin{document}

\subsection*{Описание игры}

\begin{enumerate}
    Сюжет игры разворачивается в стенах Высшей Школы Экономики, где главный герой - игрок, являющийся студентом программы «Компьютерные науки и технологии», оказывается в мире, где дисциплины, предметы и приложения превратились в агрессивных врагов, стремящихся помешать его успеваемости и сопутствующих его отчислению.\\  
    \\Цель – выжить и пройти пять уровней, соответствующих каждому из курсов обучения. Каждый уровень – это уникальное испытание, где игрок должен справляться с врагами и получать «десятки», чтобы получить доступ к следующему уровню и к финальной битве за долгожданный диплом.\\
    \\Игровой процесс построен на сочетании разнообразных препятствий. Игроку необходимо преодолеть их, используя собственные ловкость и внимательность. Чтобы уничтожить противника, игроку необходимо прыгнуть на него сверху.\\
    \\Игрок должен разрабатывать стратегию на каждом уровне, выбирая наиболее эффективный путь для достижения цели. Финальный уровень будет представлять из себя битву с боссом. Победа над ним требует применения всех приобретенных навыков управления и стратегического мышления.\\
    \\По окончании игры пользователь поймёт каково же быть студентом программы «Компьютерные науки и технологии». А концовка заставит задуматься о пройденном пути.
\end{enumerate}

\end{document}