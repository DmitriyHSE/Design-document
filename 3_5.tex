\subsection*{3.5. «Искусственный интеллект»}

Искусственный интеллект (AI) в игре разработан так, чтобы создавать атмосферу, отражающую реальные студенческие трудности, при этом оставаясь простым и понятным для игрока.

\textbf{Общие принципы поведения AI:}

\begin{enumerate}
    \item \textbf{Противники ближнего боя:}  
    Эти враги, такие как интегралы или философия, патрулируют ограниченные территории. Если игрок попадает в зону их действия, они начинают преследование. При уходе игрока за пределы зоны враги возвращаются к своему маршруту.  

    \item \textbf{Противники дальнего боя:}  
    Например, SmartLMS или Python. Эти враги атакуют, если игрок оказывается в их зоне видимости. Они фиксируются на игроке до тех пор, пока он не укрывается или не выходит за пределы зоны атаки.  

    \item \textbf{Летающие противники:}  
    Такие как вышкинская ворона или значок Яндекс.Почты. Эти враги двигаются по заданной траектории, периодически стреляя в игрока. Если игрок долго находится в их зоне, враги могут начать его преследовать.  

    \item \textbf{Поведение на уровнях:}  
    \begin{itemize}
        \item На первых уровнях враги действуют просто: патрулируют или атакуют игрока при входе в зону действия.  
        \item На более сложных уровнях враги могут действовать координированно, усиливая давление на игрока. Например, ближние и дальние противники могут перекрывать пути отхода.  
    \end{itemize}
\end{enumerate}

\textbf{Эмоциональный эффект:}

AI создаёт умеренный вызов для игрока, поддерживая атмосферу игры. Противники не только атакуют, но и добавляют комичные элементы, например, ироничные фразы или необычное поведение, что делает игровой процесс более увлекательным.