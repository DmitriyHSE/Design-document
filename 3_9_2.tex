\subsubsection{Звук и звуковые эффекты}
Звуки интерфейса обеспечивают обратную связь с игроком и делают взаимодействие с меню и элементами HUD более интуитивным. \\
Примеры звуков и их использование:
\begin{enumerate}
    \item \textbf{Звуки интерфейса}
    \begin{itemize}
        \item Клик по кнопке в меню — короткий щелчок или звенящий звук, сопровождающий выбор элемента.
        \item Подтверждение действия — мягкий звук, подчеркивающий успешное выполнение (например, старт игры или сохранение).
        \item Ошибка или недоступное действие — короткий низкий звук, информирующий игрока о невозможности выполнить действие.
        \item Сбор ресурсов (монеты, очки) — звуковой сигнал, сопровождающий изменение числовых значений на экране.
        \item Звуки паузы/возобновления игры — приглушенные или нарастающие звуки, подчеркивающие переход между игровыми и неигровыми состояниями.
    \end{itemize}
    \item \textbf{Звуки спецэффектов}
    \begin{itemize}
        \item Удар или столкновение — резкие звуки, зависящие от материала объектов, участвующих в столкновении.
        \item Смерть врага — уникальные звуки, подчеркивающие характер каждого типа врага.
    \end{itemize}
    \item \textbf{Звуки игровых объектов}
    \begin{itemize}
        \item Движение персонажа — звуки шагов с учетом поверхности.
        \item Прыжки — короткие, упругие звуки, сопровождающие момент отталкивания от земли
        \item Ловушки — звуки активации (например, щелчок механизма, свист летящего объекта или металлический лязг).
    \end{itemize}
\end{enumerate}